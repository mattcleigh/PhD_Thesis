\begin{abstract}
	This thesis forms part of the intersection of high-energy physics and machine learning, particularly for Large Hadron Collider (LHC) experiments. It underscores the synergy between collider physics and cutting-edge deep learning methods undertaken through several projects over four years. The work is thematically unified around three core concepts: graph neural networks (including transformers), deep generative models, and jets. Key contributions include developing and optimizing state-of-the-art flavour taggers for the ATLAS experiment, novel methods for reconstructing neutrino momenta using normalizing flows, and innovative approaches for forward simulation of jets employing transformer neural networks and diffusion frameworks. Furthermore, the thesis introduces a foundation model for physics data that utilizes self-supervised learning to create generalizable and meaningful representations of particle physics jets without relying on labelled data. This body of work represents a significant advancement in leveraging deep learning techniques to address the challenges posed by high-energy physics, setting the stage for future research and applications in the field.
\end{abstract}
