\chapter{Diffusion Models for Surrogate Simulation}

Developing probabilistic generative models for physics data is a challenging, yet rewarding task.
A challenge posed by the increasing data collected by collider experiments is the required computing resources for detailed simulated collisions.
Fast surrogate models can reduce the computational cost of both event and detector simulation.
These models can also be used for anomaly detection via template methods, or be used for the inverse problem of unfolding.

One of the early objectives of this thesis was to develop a generative model specifically for collider data.
However, simply reusing the highly successful models from the wider machine learning community proved insufficient.
Unlike the models used in CV, collider data is inherently point-like, and unlike to the models used in NLP, it is unordered and continuous.
These characteristics make physics data well suited to graph neural networks, hence the success of such models on classification, as discissed in ~\Cref{ch:flavour}.
These results motivated our pursuit of a generative model for physics data.


Other research groups investigated GNNs trained as GANs~\cite{MPGAN} which met success, but also included all the drawbacks of GANs as detailed in ~\Cref{ch:generative_models}.

We began work on a graph based VAE, however one of the main issues was the loss used in the reconstruction.
These modes used GNN decoders which would a point cloud of noise into the reconstructed point cloud whose transformation was conditioned on the latent space.
This worked on toy datasets, but failed at generating good samples of particle physics data.
One of the main issues was the finding a good permutation invariant loss term to use for reconstruction.
Many attempts were made, iterating on modifications to Sinkhorn~\cite{Sinkhorn}, Champfer~\cite{Chamfer}, and optimal transport losses.
Even physically motivated metrics specific to collider data~\cite{MetricOfCollider}, yeilded poor results.

During this time, the diffusion models started to gain traction in the wider machine learning community, and a realization was made that this could be the key to solving the problem.
A key insight was that the training objective in diffusion training is not full generation, but denoising.
Given set of nodes each individually corrupted with noise, then training the model to remove the noise is a permutation equivariant task and thus could be done with simply MSE as the loss.
This led us to the first successful diffusion model for generating particle type data.

This chapter details a collection published of work used to develop models for the conditional generation point cloud data, including particle physics jets~\cite{PCJedi, EpicJedi, PCDroid} and full events~\cite{PIPPIN}.
It also covers the use of these models for template based anomaly detection~\cite{Drapes}.


\section{Introduction}

\section{PC-JeDi}

\section{PC-Droid}

% \section

\section{Anomaly Detection}

\subsection{Introduction}

