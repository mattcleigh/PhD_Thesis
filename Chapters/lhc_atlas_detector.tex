
\chapter{The Large Hadron Collider and the ATLAS Detector}
\label{ch:lhc_atlas_detector}

The Large Hadron Collider (LHC)~\cite{LHCMachine,LHC1,LHC2, LHC3} is a synchrotron collider located at the Conseil Européen pour la Recherche Nucléaire (CERN) near Geneva, Switzerland.
It is also the world's most powerful particle accelerator, providing both proton-proton ($pp$) and heavy ion collisions at the high energy frontier.
At maximum speed, the protons in the LHC are only $3~\mps$ slower than the speed of light.

Four primary experiments utilize the $\TeV$ scale collisions generated the each of the the four interaction points (IPs) of the LHC.
Two general-purpose experiments, ATLAS~\cite{ATLAS} and CMS~\cite{CMS}, employ hermetic detectors to study various physics processes.
The LHCb experiment~\cite{LHCb} focuses on measuring the properties of B mesons and other particles containing b-quarks, which are significant for understanding CP violation.
The Alice experiment~\cite{ALICE} investigates the properties of quark-gluon plasma, a state of matter believed to have existed in the early universe.
Additionally, the LHC hosts several smaller, highly specialized experiments including LHCf~\cite{LHCf}, TOTEM~\cite{TOTEM}, MoEDAL~\cite{MoEDAL}, FASER~\cite{FASER}, and SND@LHC~\cite{SNDLHC}.
All LHC experiments strive to answer fundamental questions about the universe; to test the predictions of the Standard Model of particle physics or to search for new physics beyond it.

This section describes common collider physics definitions, such as the coordinate system and luminosity,
the LHC along with its injector chain, and the ATLAS experiment.
As the focus of this work is on data and tools for $pp$ collisions at the LHC, the discussion is limited to this context.
Additionally, the components of this thesis which utilized official ATLAS simulation and software, \Cref{ch:spice}, were limited to tracks and jets.
Therefore, the information provided here is focused on these topics only.

\section{The Large Hadron Collider}

The LHC is a somewhat circular collider with a circumference of $26.7~\km$.
It is closer to an irregular octagon with rounded corners and approximately $ 530~\m $ long straight sections, housed within a tunnel whose depth varies between $45~\m$ and $170~\m$ underground.

The machine consists of two ring acceleration pipes, each carrying a separate beam of protons or heavy ions that are accelerated in opposite directions.
The beams are surrounded by 1232 superconducting niobium-titanium (NbTi) dipole magnets that generate an $8.33~\tesla$ magnetic field to bend the protons around the curved sections of the ring.
Many more higher-order multipole magnets are used for beam insertion, cleaning, dumping, and focusing towards the experiment IPs.
Along one of the straight sections, 16 superconducting radiofrequency cavities, eight per beam, operate at $400~\MHz$ to accelerate the protons to their maximum energy.
The LHC was designed to deliver a centre-of-mass energy of $\sqrt{s}=14~\TeV$ but has yet to reach this due to technical limitations.

The beams are not continuous streams of protons but are instead segmented into bunches, each containing on the order of $10^{11}$ particles.
The radiofrequency cavities are tuned to ensure that these bunches are tightly packed, each as long as $7.5~\cm$ and separated by only $25~\ns$.
For collisions to occur, the bunches from the two opposing beams are compressed to a space of $64~\um$, around the width of a strand of hair, and made to pass through each other.
Even with such compression and more than a hundred billion protons in each bunch, the average number of interactions per bunch crossing \avemu is less than 100.

\subsection{The CERN Accelerator Complex}

The LHC is a synchrotron, meaning that it cannot accelerate particles from rest.
Therefore, it is only the final stage of the CERN accelerator complex, which is shown in \Cref{fig:cern_accelerator_complex}.
Each stage in the chain of accelerators increases the energy of the particles within before passing them to the next stage.
Before 2020, the chain began with the Linear Accelerator (Linac) 2, which accelerated protons to $50~\MeV$.
The input to Linac2 was hydrogen gas, stripped of its electrons by a strong electric field.
Following Run 2, the chain was updated to begin with Linac4, which operates on negative hydrogen ions, not protons.
This change reduces the beam loss, allowing more particles to accumulate in the next stage.
Furthermore, Linac4 introduce a three-fold increase in beam energy, outputting particles at $160~\MeV$.
The Linacs feed the Proton Synchrotron Booster (PSB), which increases the energy to $2~\GeV$ and also serves to strip the electrons from the hydrogen ions.
From the PSB, the protons are passed to the Proton Synchrotron (PS), which accelerates them to $26~\GeV$, and then to the Super Proton Synchrotron (SPS), which accelerates them to $450~\GeV$.
From there, they are injected into the LHC.
This entire process takes a few seconds, but several minutes are required to fill the entire bunch train of the LHC.
The LHC then accelerates the bunches to the final energy, which, as of Run 3, is $6.8~\TeV$ per beam.

\begin{figure}
    \centering
    \includegraphics[width=0.99\textwidth]{Figures/cern_atlas/complex.png}
    \caption{The layout of the CERN accelerator complex as of 2022~\cite{CERNAC}.}
    \label{fig:cern_accelerator_complex}
\end{figure}

\subsection{Coordinate Systems}

In collider experiments, the typical coordinate system places the origin at the nominal IP, which, for hermetic detectors like ATLAS, Alice, and CMS, is at the centre of the detector.
The coordinate systems used by these experiments are right-handed, as shown in \Cref{fig:atlas_coordinate_system}, with the $z$-axis points along the beamline, the $y$-axis pointing upwards, and the $x$-axis pointing towards the centre of the LHC ring.
The transverse plane is defined as the plane perpendicular to the beam axis and is spanned by the $x$ and $y$ coordinates.

A polar representation is also often used, where the azimuthal angle $\phi$ is measured in the transverse plane from the $x$-axis, and the polar angle $\theta$ is measured from the $z$-axis.
The polar angle is frequently replaced with pseudorapidity $\eta$, defined as
\begin{equation}
    \eta = -\ln \left( \tan \left( \frac{\theta}{2} \right) \right).
\end{equation}
Differences in pseudorapidity are Lorentz invariant under boosts in the direction of the beamline, making it a useful quantity for describing the products of high-energy collisions.

Angular distances between objects are often measured in $\Delta R = \sqrt{(\Delta \eta)^2 + (\Delta \phi)^2}$, where $\Delta \eta$ and $\Delta \phi$ are the differences in pseudorapidity and azimuthal angle.
For collision products, a key observable is a transverse momentum $\pt=\sqrt{p_x^2 + p_y^2}$, where $p_x$ and $p_y$ are the components of the momentum in the transverse plane.

\begin{figure}
    \centering
    \includegraphics[width=0.6\textwidth]{Feynman/coordinate.pdf}
    \caption{The right-handed coordinate system used by the ATLAS experiment.}
    \label{fig:atlas_coordinate_system}
\end{figure}

\subsection{Cross-Section and Luminosity}

The cross-section, denoted by $\sigma$, indicates the likelihood of an interaction between two colliding particles.
Although calculating the cross-section involves quantum mechanical transition matrix elements and phase space integrals, it can be likened to the effective cross-sectional area of the particles participating in the interaction.

The total number of interactions $N$ depends on the cross-section of the process $\sigma$ and the integrated luminosity $L_{\text{int}}$ by the formula,
\begin{equation}
    N = \sigma L_{\text{int}} = \sigma \int \mathcal{L}(t) dt,
\end{equation}
where $\mathcal{L}(t)$ is the instantaneous luminosity.
This measures how tightly packed colliding particles are in a unit of space and time.
For two colliding beams with Gaussian densities, the instantaneous luminosity is given by
\begin{equation}
    \mathcal{L}(t) = \frac{N_1 N_2 f}{4 \pi \sigma_x \sigma_y} S(\theta_c),
\end{equation}
where $N_1$ and $N_2$ are the number of particles in each beam, $f$ is the frequency of bunch crossings, $\sigma_x$ and $\sigma_y$ are the root-mean-square beam widths in the $x$ and $y$ directions, and $S(\theta_c)$ is the geometric reduction factor due to the crossing angle $\theta_c$.
This last factor exists because the beams do not collide head-on, as this would result in multiple collision points, so they are crossed at a finite angle $\theta_c$.
This reduces the overlap of the beams when they pass through one another, reducing the effective luminosity.
For the $pp$ collisions at the LHC $S(\theta_c)$ is approximately $0.61$.
Even with this finite angle, long-range beam-beam interactions can still occur and must be accounted for.

The LHC was designed to deliver $\mathcal{L}(t) = 10^{34}~\unit{\centi\meter^{-2}\second^{-1}}$ at the IPs of ATLAS and CMS.
However, this has not been constant over the years, and each experiment independently measures the luminosity using well-understood processes.
The total integrated luminosity delivered by the LHC and recorded by ATLAS is shown in \Cref{fig:luminosity}.
The LHC's instantaneous luminosity has increased with each run and this is important as it boosts the production rate of all physical processes.
It particularly beneficial for measurements on rare events where the uncertainty is statistically limited.

\begin{figure}
    \centering
    \includegraphics[width=0.32\textwidth]{Figures/cern_atlas/lumi1.pdf}
    \includegraphics[width=0.32\textwidth]{Figures/cern_atlas/lumi2.pdf}
    \includegraphics[width=0.32\textwidth]{Figures/cern_atlas/lumi3.pdf}
    \caption{The total integrated luminosity delivered by the LHC and recorded by the ATLAS experiment for Run 1 (left)~\cite{run1data}, Run 2 (middle)~\cite{run2data}, and Run 3 (right)~\cite{run3data}.}
    \label{fig:luminosity}
\end{figure}

\subsection{Pileup}

The increase in luminosity between Run 1 and Run 3 was in part achieved by raising the number of protons per bunch.
This in turn leads to more particle interactions per bunch crossing.
One of the undesirable effects of this is the large amount of overlapping signals in the detector which can degrade the resolution and efficiency of reconstruction algorithms.

Each bunch crossing in the collider generates a distinct event for analysis.
While the average number of interactions per crossing has increased with each LHC run, \Cref{fig:pileup} shows a wide distribution of interaction counts within each run.
Generally, only one of these interactions -- the hard scatter -- produces the high-energy particles captured by the detector and thus is of primary interest.
The remaining ``soft scatters'' are low-energy interactions that are known as in-time pileup.
Detector materials have a response and reset time, which can lead to residual signals from previous crossing being rerecorded, this is known as out-of-time pileup.
Filtering out signals from all pileup sources is crucial for accurate data analysis and poses a significant challenge as the LHC's luminosity increases.

\begin{figure}
    \centering
    \includegraphics[width=0.6\textwidth]{Figures/cern_atlas/mu_run123.pdf}
    \caption{The distribution of the number of interactions per bunch crossing over the three LHC runs as measured by the ATLAS experiment~\cite{run3data}.}
    \label{fig:pileup}
\end{figure}

\subsection{LHC Runs}

The LHC operates on multi-year runs, the first beginning in 2010.
In between each run is a period of \textit{long-shutdown}, allowing for maintenance and upgrades to both the accelerator and the experiments.
The various run properties are shown in \Cref{tab:lhc_runs}.
Due to an electrical fault damaging many of the superconducting magnets in 2008, Run 1 was much delayed and operated at a reduced $\sqrt{s}=7~\TeV$ using a sparse bunch structure, only colliding at $20~\MHz$.
In 2012, the energy was increased to $\sqrt{s}=8~\TeV$.
The total delivered luminosity of the run was $28.4~\ifb$.
Run 2 spanned from 2015 to 2018, with an increased energy of $\sqrt{s}=13~\TeV$, a bunch crossing rate of $40~\MHz$, and a total integrated luminosity of $156~\ifb$.
Run 3 began in 2022 and is expected to continue until 2026 and has already surpassed Run 2 with $195~\ifb$ of data delivered~\cite{run3data}.

\begin{table}[h!]
    \centering
    \caption{Overview of the different LHC runs~\cite{LHCRun2,LHCRun3,run1data, run2data, run3data}. Values for Run 3 are preliminary as data is still being collected.}
    \label{tab:lhc_runs}
    \resizebox{\textwidth}{!}{
        \begin{tabular}{lcccccc}
            \toprule
                              & $\sqrt{s}~[\TeV]$ & $\mathcal{L}_\text{int}~[\ifb]$ & Protons per bunch     & Bunch spacing $[\ns]$ & \avemu \\
            \midrule
            Run 1 (2010-2011) & 7                 & 5.5                             & $1.45 \times 10^{11}$ & 50                    & 10     \\
            Run 1 (2012)      & 8                 & 22.8                            & $1.6 \times 10^{11}$  & 50                    & 21     \\
            Run 2 (2015-2028) & 13                & 156                             & $1.25 \times 10^{11}$ & 25                    & 34     \\
            Run 3 (2022-)     & 13.6              & 195                             & $1.8 \times 10^{11}$  & 25                    & 50     \\
            \bottomrule
        \end{tabular}
    }
\end{table}

\section{The ATLAS Experiment}

The ATLAS experiment~\cite{ATLAS, ATLASRun3} is one of the four primary experiments at the LHC\@.
It is the largest particle detector of the four, weighing $7000$ tonnes and measuring $44$ m long and $25$ m in diameter.
It is a general-purpose detector designed to explore a wide range of physics phenomena.
To achieve this, it has multiple sub-detectors designed to capture a broad range of outgoing products from the $pp$ collisions.
The ATLAS detector is shown in \Cref{fig:atlas_detector}.

The detector is hermetic, meaning it covers nearly the entire solid angle of $4\pi$.
It is arranged in concentric cylindrical-like layers around the IP\@.
Each layer contains a barrel region, which wraps around the beamline, and end-cap regions, which cover the flat edges of the cylinder on either side.

This section describes the ATLAS detector, focusing on the tracking systems.
These systems provide precise position and momentum measurements of charged particles, which are used to reconstruct particle trajectories and vertices.

\begin{figure}[htb]
    \centering
    \includegraphics[width=0.99\textwidth]{Figures/cern_atlas/ATLAS.png}
    \caption{The ATLAS detector at the LHC\@. Image taken from Ref.~\cite{ATLASRun3}.}
    \label{fig:atlas_detector}
\end{figure}

\subsection{Inner Detector}

The Inner Detector (ID) is the innermost sub-detector of ATLAS and is shown in \Cref{fig:atlas_inner_detector}.
It comprises three sub-detectors: the Pixel Detector, the Semi-Conductor Tracker (SCT), and the Transition Radiation Tracker (TRT).
It is designed to capture charged particle signals with a pseudorapidity $|\eta| < 2.5$.
These signals are clustered into points in space where the particles have ionized the detector material.
These points can be used to reconstruct the charged particle's trajectory -- track -- as it propagates out from the centre of the detector.

The ID is immersed in a $2~\tesla$ axial magnetic field generated by a 5.8 m long superconducting solenoid magnet containing over $9\km$ of NbTi wire.
The magnetic field causes the particles to bend in a plane perpendicular to the beamline, tracing out a helical path whose curvature is proportional to $q/p$, where $q$ is the charge of the particle and $p$ is the magnitude of its momentum.

A fully reconstructed track is characterized by five parameters, which are shown in \Cref{fig:track_parameters}.
These include the charge-to-momentum ratio $q/p$, the azimuthal angle $\phi$, the polar angle $\theta$.
By tracing the track back to the beamline, the longitudinal $z_0$ and transverse $d_0$ impact parameters are determined at the point of closest approach to the IP\@.
These impact parameters represent the distances measured in each respective direction.
The relative $\pt$ resolution of the ID is described as,
\begin{equation}
    \sigma(\frac{1}{\pt}) = 0.36 \oplus \frac{13}{\pt \sin(\theta)}~\TeV
\end{equation}
where $\oplus$ denotes a sum in quadrature.

\begin{figure}[htb]
    \centering
    \includegraphics[width=0.6\textwidth]{Figures/cern_atlas/Track.png}
    \caption{The main parameters of a reconstructed track in the ATLAS Inner Detector. Image taken from Ref.~\cite{ATLASTrackingSoftware}.}
    \label{fig:track_parameters}
\end{figure}

\begin{figure}[htb]
    \centering
    \includegraphics[width=0.59\textwidth]{Figures/cern_atlas/BetterID.png}
    \includegraphics[width=0.39\textwidth]{Figures/cern_atlas/IDCut.png}
    \caption{Schematic of the full ATLAS Inner Detector (left) and a cross-section of the barrel region (right). Images taken from Refs.~\cite{ATLASRun3,IBLPhotos}.}
    \label{fig:atlas_inner_detector}

\end{figure}

\subsubsection{Pixel Detector}

The high-granularity silicon pixel detector~\cite{ATLASPixel} covers the innermost region of the ID.
It measures 1.4 m long and is fully contained in a diameter of 0.5 m.
High granularity is required at this scale to resolve the numerous tracks that pass through it.
Individual silicon pixels are about $50~\um$ wide in the transverse direction and about $400~\um$ long in the longitudinal direction.

Each pixel has a slight potential difference applied across it.
Ionizing radiation left by a traversing charged particle generates electron-hole pairs that drift and are collected at the p-n junctions, providing a signal.
The pixel detector has 1744 modules, each containing around 46k pixels, resulting in 80M readout channels.
These are arranged in three barrel layers placed at a radius of $50.5~\mm$, $88.5~\mm$, and $122.5~\mm$ from the beamline, with four end-cap disks on each side.

Between Run 1 and 2, an additional layer was added at a radius of $33~\mm$ to the beamline, known as the Insertable B-Layer (IBL)~\cite{ATLASIBL}.
Pixels in the IBL measure $50~\um$ in the transverse direction and $250~\um$ in the longitudinal direction.
This layer improved the $\pt$ resolution of the detector by close to 30\% for low $\pt$ tracks.
Furthermore, the enhanced $z$-resolution improved impact parameter resolution and cluster separation, allowing for better reconstruction of primary and secondary vertices.

\subsubsection{Semiconductor Tracker}

The Semiconductor Tracker (SCT)~\cite{ATLASSCT} surrounds the Pixel Detector, occupying the region between $299~\mm$ and $514~\mm$ from the beamline.
It comprises four concentric barrel layers with nine disks at each end-cap covering the region $|\eta| < 2.5$.
Like the pixel detector, the SCT is composed of silicon readout channels. Unlike the pixels, each microstrip measures $20~\um \times 12~\cm$ and thus only has high granularity in a single direction, oriented to the transverse momentum.
To improve longitudinal resolution, each SCT layer includes two back-to-back microstrip layers, rotated by $40 \unit{\milli\radian}$ relative to each other.
The SCT in total has around 6.3M readout channels.

\subsubsection{TRT}

The outermost sub-detector of the ID, occupying the volume from $554~\mm$ to $1082~\mm$ from the beamline, is the Transition Radiation Tracker (TRT)~\cite{ATLASTRT}.
It is a straw-tube tracker that aids in the momentum measurement and provides information for particle identification for tracks within $|\eta| < 2.0$.
The TRT provides the most hits per track, around 35 on average, of the three sub-detectors.
At a greater distance from the beamline, it also does not require such fine spatial resolution as the silicon detectors to provide a good measurement of the track kinematics.

It comprises approximately 300k thin-walled drift tubes, each $4~\mm$ in diameter
The barrel region contains 50k tubes, each $144~\cm$ long, while the end caps contain 250k tubes, each $37~\cm$ long.
A potential difference exists between the tube walls and an axial wire.
When a charged particle intersects the tube, it ionizes the predominantly argon gas inside, and the resulting electrons drift towards the wire, providing the signal.
There is no information on where the ionization occurred along the length of the tube.
Thus, resolution for the barrel region is only defined in the $r-\phi$ plane, and for the end-cap, it is only defined for $z-\phi$.
A polymer foil is placed between the tubes to generate transition radiation, the pattern of which allows discriminating between electrons and pions.

\subsection{Calorimeters}

ATLAS has three calorimeter systems outside the central solenoid magnet, which measure the energies of particles produced by the collisions.
Each calorimeter is designed with alternating layers of absorber and active materials.
Particles interact with the absorbers, lose energy, and produce showers of secondary particles that ionize the active material, which can be converted into a measurable signal.
These layers are typically arranged in an accordion-like structure to ensure homogenous coverage.
Different absorber materials are used to optimize the interaction and thus containment of the showers for different particle types.

The Electromagnetic Calorimeter (ECal) is specialized for measuring the energies of electrons and photons.
Surrounding this is the Hadronic Calorimeter (HCal), which is designed to measure the energies of hadrons.
These subsystems cover the region $|\eta| < 3.2$.
In addition, there is also the forward calorimeter subsystem, which measures the energy of particles in the range $3.2 < |\eta| < 4.9$.
However, information from this subsystem is not used in this work, and it is omitted from the following descriptions.
A schematic of the calorimeter systems is shown in \Cref{fig:atlas_calorimeters}.

\begin{figure}[htb]
    \centering
    \includegraphics[width=0.99\textwidth]{Figures/cern_atlas/Calos.png}
    \caption{Schematic of the ATLAS calorimeter systems. Image taken from Ref.~\cite{ATLASRun3}.}
    \label{fig:atlas_calorimeters}
\end{figure}

\subsubsection{Electromagnetic Calorimeter}

The Cal~\cite{ATLASECal} is a high granularity calorimeter which uses liquid argon as the active material and lead as the absorber.
To keep the argon in a liquid state, the ECal is cooled to $-185~\unit{\degree C}$.
These materials are chosen to optimize the containment of electromagnetic showers produced by electrons and photons, via pair production and bremsstrahlung radiation, while minimising the energy loss of heavier particles.
The barrel ECal covers the range $|\eta| < 1.475$ and is split into three layers, decreasing in granularity with distance from the beamline.
The inner layer allows for more precise shower measurements which assists in the identification of pions.
The Liquid Argon Electromagnetic Enc-Cape (EMEC) and covers the range $1.375 < |\eta| < 3.2$.
Unlike the ID, the ECal resolution increases with incident particle energy.

\subsubsection{Hadronic Calorimeter}

The HCal or Tile Calorimeter~\cite{ATLASHCal} encompasses the ECal and is designed to measure the energies of hadrons.
It uses steel as the absorber and scintillating tiles as the active material made from polystyrene.
The hadrons are contained by losing energy through inelastic nuclear interactions with the steel, producing a shower of electrons and photons which are detected by the tiles.
In the forward regions of the calorimeter, $1.5 < |\eta| < 3.2$, a copper/LAr combination is used.
The granularity and energy resolution of the HCal are lower than the ECal.

\subsection{Muon Spectrometer}

Other than neutrinos, which traverse the entire detector without interacting, muons are the only particles that easily penetrate through the calorimeters, incurring minimal energy loss.
Though they leave tracks in the ID, the Muon Spectrometer (MS)~\cite{ATLASMuon,ATLASRun3} provides triggering information, identification, and extra momentum measurements for muons.
The MS is the outermost sub-detector of ATLAS and is shown in \Cref{fig:atlas_muon_spectrometer}.
Its systems are arranged in three barrel layers covering $|\eta| < 1.2$ and several end-cap layers or wheels covering $1 < |\eta| < 2.7$.
The wheels are placed on either side of the detector at a distance of $7.4$~m, $14$~m, and $21.5$~m from the IP.

Monitored Drift Tubes are the spectrometer's primary tracking component and are similar in principle to the straw tubes in the TRT@.
More than 380k aluminum tubes, each with a diameter of $30$~mm and primarily filled with argon gas, are organized into three barrel layers and four end-cap disks.
In addition, Cathode Strip Chambers are used in inner layers of the end-cap to cope with the higher particle flux.
These are multi-wire proportional chambers built into strips.

Three large air-core toroidal magnets generate the magnetic field in the MS\@, one in the barrel region and one in each end-cap regions.
Each magnet has eight coils and deflects the muons, allowing for the measurement of their momentum.

\begin{figure}[htb]
    \centering
    \includegraphics[width=0.99\textwidth]{Figures/cern_atlas/MS.png}
    \caption{Schematic of the ATLAS Muon Spectrometer. Image taken from Ref.~\cite{ATLASRun3}.}
    \label{fig:atlas_muon_spectrometer}
\end{figure}

\section{Reconstruction}
\label{sec:event_reconstruction}

Reconstruction is the process of converting the raw detector signals into collections of defined physical objects to better understand the collision events.
This process is performed offline, after the data has been recorded and stored, and uses information from all sub-detectors.

This section describes the basics of track, vertex, and jet reconstruction at ATLAS, as well as several approaches to jet tagging.

\subsection{Track Reconstruction}

Track reconstruction refers to taking the collection of hits registered by the ID, and grouping them into individual tracks likely to have been produced by a single charged particle.
Tracks are key ingredients in the reconstruction and identification of other particles, such as electrons and muons, jets, and the multiple vertices produced in each collision.
A full description of the track reconstruction pipeline used for ATLAS Run 2 data can be found in Ref.~\cite{PerformanceATLASTrack}.
Track reconstruction is performed in nuerous stages.
Hits are first clustered into space points, after which iterative track-finding algorithms are used, followed by cleaning and ambiguity resolution.

\subsubsection{Clusterization}

A single particle may deposit energy in multiple adjacent pixels or strips in the ID\@.
This is due to the drift of electrons between sensors within the solenoid magnetic field or from the incident angle of the particle.
These hits are first combined into clusters which are then used to form \textit{space points} (SP) defined in three dimensions.
For the SCT, these clusters also combine information from the two back-to-back layers.
Position uncertainties of these points are estimated from cluster sizes and detector geometry.
In dense environments the clustering algorithm may merge mistakenly merge hits from different particles, leading to a higher rate of fake tracks.

\subsubsection{Track Seeding}

Tracks are \textit{seeded} by grouping triplets of SPs in three sequential layers in either the pixel or SCT subdetectors.
Each of these seeds forms a \textit{track candidate} and are used to determine the first crude estimates of the track parameters, assuming a perfectly uniform magnetic field and helical trajectory.
At this stage, hits may be shared between multiple seeds, and the seeds are not required to be consistent with the primary vertex.
This maximizes the number of possible seeds and thus the efficiency.
The only requirement to improve purity is that one additional SP is found compatible with the track candidate outside the seeding region.
The following steps are therefore designed to improve the track quality and remove fake tracks.

\subsubsection{Ambiguity Resolution}



% The trajectory of a charged particle (track) is parameterized as a helix using five parameters:

% Impact parameter significances are calculated as:

% In flavor tagging, IP significances are lifetime signed:

% \begin{itemize}
%     \item $d_0$ and $z_0$: Transverse and longitudinal impact parameters (IP), specifying the closest approach of the track to the primary vertex.
%     \item $\phi$ and $\theta$: Azimuthal and polar angles.
%     \item $q/p$: Measured charge divided by the scalar 3-momentum.
%     \item $s(d_0) = \frac{d_0}{\sigma(d_0)}$
%     \item $s(z_0) = \frac{z_0}{\sigma(z_0)}$
%     \item Positive: Track crosses the jet axis in front of the primary vertex.
%     \item Negative: Track crosses the jet axis behind the primary vertex.
% \end{itemize}

% \subsection{Jets}

% For example, the differentiable cross-section of the ($2\rightarrow n+1$) process in \Cref{fig:quark_gluon} in the extreme soft and collinear limits factorizes into a $2 \rightarrow n$ process times a $1 \rightarrow 2$ split.
% In this extreme limit the probability for soft-colinear gluon emission is given by,
% \begin{equation}
%     \label{eq:gluon_emission}
%     d \mathcal{S} = \frac{2 \alpha_s C_F}{\pi} \frac{d \theta}{\sin\theta} \frac{dE}{E} \frac{d\phi}{2\pi},
% \end{equation}
% where $C_F = \sfrac{4}{3}$ is the color factor for the quarks, $\theta$ is the angle of the emitted gluon with respect to the quark, $E$ is the energy of the emitted gluon, and $\phi$ is the azimuthal angle.

% \subsubsection{Particle Flow}
% \label{sec:particle_flow}

% \subsubsection{Fat Jets}
% \label{sec:fat_jets}

% At the LHC, quarks can be produced in the decays of particles such as $W/Z$~bosons, or top quarks through their decay into a $W$~boson and a $b$-quark.
% For most energy scales, the two or three quarks from these decays produce jets that can be individually resolved in the detector.
% However, as the momenta of the intermediate particles increase, the decay products themselves start to collimate, resulting in a single large-radius or \textit{fat} jet in the detector; this is the so-called boosted regime.
% These objects are colloquially referred to as \textit{fat jets}.
% The vast majority of jets, however, are initiated by partons, which are not the decay products of other massive particles (QCD background).
% https://tikz.net/jet_top/

% \subsection{Jet substructure}
% \label{sec:jet_substructure}
% Properties relating to the distribution of constituents within a jet are known as the substructure of a jet, which can be used to identify the original seed particle~\cite{Kogler:2018hem}.
% This is particularly interesting in the boosted regime, where several partons from the decay of another elementary particle have overlapping showers.
% % In boosted topologies it is of particular interest to identify whether a jet is from hadronic decays of a resonant particle or otherwise.
% % Therefore, accurate modelling of jet substructure is of importance in many measurements and searches at the LHC.
% % Many high level features are defined based on physics principles and are calculated from the constituents in a jet.

% One commonly used set of observables to describe jet substructure is its \emph{N-subjettiness}~\cite{Thaler_2011}, denoted by ${\tau_N}$.
% These are useful in identifying jets originating from processes with $N$ prongs as a result of the decay of the initial particle.
% A jet originating from a gluon is likely to have a 1-prong substructure, whereas a $W/Z$~boson decay is likely to produce a 2-prong jet, and a jet originating from the all-hadronic decay of a top quark will tend to be 3-prong.
% Other commonly used observables relate to the energy correlation functions of a jet and their ratios, such as $\Dtwo$~\cite{Larkoski_2013,Larkoski_2014}.%
% \footnote{$\Dtwo$ is defined as the ratio of the three-point and cubed two-point energy correlation functions.}
% A new set of features which have been found to be sensitive to the underlying substructure of different jet types are the Energy Flow Polynomials~(EFPs)~\cite{Komiske_2018}.

% Furthermore, when a seed particle decays, the observed opening angle of the decay products is strongly dependent on its mass and momentum.
% In jets, this means that the distribution of constituent properties is strongly correlated to the overall invariant mass and transverse momentum \pt.

% Classification approaches applying cuts on the substructure features, as well as machine learning algorithms trained using such features have been successfully employed in the ATLAS and CMS collaborations to distinguish jets originating from $W$~bosons ($W$-jets), top quarks (top jets), gluons, and light quarks~\cite{ATLAS:2018wis,CMS:2020poo}.
% In recent years, more sophisticated classification algorithms have been trained on the constituents themselves, either represented as ordered vectors~\cite{pearkes2017jet,ATLAS:2018wis,CMS:2020poo,Butter_2018}, images~\cite{de_Oliveira_2016,Kasieczka_2017,Macaluso_2018}, or point clouds~\cite{ParticleNet,Komiske:2018cqr,Moreno:2019bmu,Dreyer:2020brq,Dolan:2020qkr,Mikuni:2021pou,Shimmin:2021pkm,Gong:2022lye,Qu:2022mxj} (see Ref.~\cite{TopLandscape} for a review).

% These approaches are very sensitive to the substructure of jets originating from different particles.
% As such, when using fast surrogate models it is crucial that they accurately capture the distribution of the constituents within a jet and their correlations to the mass and \pt.

% \section{Event Simulation}

% \section{Flavour Tagging}
% \label{sec:flavour_tagging}

% A diagram of the GN1 model is shown in \Cref{fig:gn1}.

% \begin{figure}
%     \centering
%     \includegraphics[width=0.99\textwidth]{figures/atlas/gn1.png}
%     \caption{Schematic diagram of the GN1 model from Ref.~\cite{GN1}.}
%     \label{fig:gn1}
% \end{figure}
